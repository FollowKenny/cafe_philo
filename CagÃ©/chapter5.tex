\documentclass[10pt,t]{beamer}
\usetheme{Antibes}

\usepackage[utf8]{inputenc}
\usepackage[T1]{fontenc}
\usepackage{eurosym}

\title{Le prix de la démocratie}
\subtitle{Un espoir le financement des partis et des campagnes électorales}
\author{Ivan}
\date{Décembre 2020}


\begin{document}
% Title
\frame{\titlepage}

% TOC
\begin{frame}{Overview}
    \tableofcontents
\end{frame}

\AtBeginSection[]
  {
     \begin{frame}<beamer>
     \frametitle{Plan}
     \tableofcontents[currentsection]
     \end{frame}
  }

% Section 1 : contenu du chapitre et objectifs
\section{Introduction}
\begin{frame}{Introduction}
    \begin{itemize}
        \item \alert{Objectif du chapitre} : Faire le point sur les financements publics des partis en occident dans l'histoire récente et comparer les systèmes entre eux.
        \item \alert{Thèse} : Menace actuel sur les financements publics $\rightarrow$ recours aux financements privés $\rightarrow$ augmentation des inégalités et populismes.
        \item \alert{Volonté} : Même montant d'argent publique pour les préférences de chaque citoyen et fluidité dans l'expression des préférences (fréquence, émergence de partis\dots).
        \item \alert{Constats} : 
            \begin{itemize}
                \item Il n'existe pas de bon système parmi ceux étudiés (Angleterre, USA, Canada, Italie, Espagne, Allemagne, France).
                \item Ni même de modèle européen vs modèle US.
                \item Actuellement l'argent privé est en train de gagner la bataille et les USA sont plutôt une photo du futur au mieux.
            \end{itemize} 
    \end{itemize}
\end{frame}

% Section 2 : USA
\section{USA}
\begin{frame}{les fondements}
    \begin{itemize}
        \item \alert{1907 Roosvelt Tillman Act} : nécessité de financement public des campagnes et limitations des dons privés. Obtiens le second en 1907 pas le premier qui doit arriver 50 ans plus tard.
        \item \alert{1970 FECA et Revenue Act} : Mise en place du Presidential Fund. Mesures comparables 1960 Allemagne, 90 pour la France et la Belgique
    \end{itemize}
\end{frame}

\begin{frame}{Presidential Fund}
    \begin{itemize}
        \item \alert{Alimentation} : Cocher une case sur la déclaration d'impôts si on veut que 3 dollars aillent à ce fond (dépense supplémentaire pour l'état fédéral pas pour le citoyen).
        \item \alert{Chiffres} : 1974 85M\euro\ 0.61\euro\ pa. 1978 93M\euro\
        \item \alert{Dépense} : 
            \begin{itemize}
                \item Campagnes des primaires par un système de matchin 1\$ privé = 1\$ public avec limite à 250\$ pour favoriser les petits donateurs.
                \item Conventions nationales des partis. Les minor parties (5-25 \% des voix) reçoivent des fonds en proportion du \% de suffrage reçu (1-1).
                \item Campagne présidentielle par une donation égale aux deux principaux candidats = au plafond des dépenses électorales (0.41\euro\ $\rightarrow$ 0.31\euro\ ). Pour pouvoir en bénéficier $\rightarrow$ limite ses dépenses au plafond et refuser l'argent privé. 
            \end{itemize} 
    \end{itemize}
\end{frame}

\begin{frame}{Evolution du financement public}
    \begin{itemize}
        \item drop depuis 2000
        \item \alert{2006} : Première election Obama renonce au financement public.
        \item \alert{2012} : Plus de financement public sauf les CNP.
        \item \alert{2016} : Juste plus.
        \item Absolument pas lié à un manque de moyen de l'état fédéral mais c'est un choix des politiciens.
        \item lol depuis le PF ne cesse de gonfler.
    \end{itemize}
\end{frame}

\begin{frame}{Le cas du local}
    \begin{itemize}
        \item \alert{Pas de financement systématique} $\rightarrow$ choix des états. Actuellement 13 le fond.
        \item \alert{Plusieurs systèmes}. A noter Seattle 2017 chèques démocratiques. Par contre toujours la même limite (plafon ou full privé).
    \end{itemize}
\end{frame}

\begin{frame}{Synthèse}
    \begin{itemize}
        \item Précuseur
        \item Dead now (rien en local, rien au partir, négligeable aux élections)
    \end{itemize}
\end{frame}

% Section 3
\section{Financement public : une réponse aux scandales - étude comparative}
\begin{frame}{Exemples rapides}
    \begin{itemize}
        \item \alert{USA} : En même temps que le watergate.
        \item \alert{Italie} : 1974 en même temps qu'un scandale de corruption des partis.
    \end{itemize}
\end{frame}

\begin{frame}{Canada}
    \begin{itemize}
        \item \alert{1964} : Commission Barbeau dans un contexte de scandales financiers $\rightarrow$ 10 ans plus tard financement public.
        \item \alert{Principes} :
            \begin{itemize}
                \item Remboursement d'une fraction des dépenses des partis.
                \item Remboursement de la moitié des dépenses de campagne si +15\% des voix
                \item Defiscalisation des dons.
                \item Plafond de dépenses.
            \end{itemize}
        \item \alert{2004} : Nouvelles évolutions Federal Electoral Reform sur fond de scandale financier entre agences de com et parti libéral $\rightarrow$ introduction d'un financement direct des partis.
        \item \alert{Montants} : $\approx$ 0.75\euro\ pa jusqu'en 2011 puis drop puis suppression totale en 2016..
        \item Au final en 2017 : - de 0.25\euro\ pa.
    \end{itemize}
\end{frame}

\begin{frame}{France}
    \begin{itemize}
        \item \alert{Tardivement mis en place} : 1988-90. Fais suite à de nombreuses affaires de financement occulte.
        \item \alert{Principes} : Deux parts égales, selon les résultats aux dernières législatives, selon le nombre de parlementaires par partis chaque année $\rightarrow$ financement figé pour cinq ans $\rightarrow$ comment faire émerger un mouvement ?
        \item \alert{Chiffres} : 63M\euro\ (1\euro\ pa) en 2017 contre 100M\euro\ en 94, en décroissance linéaire par morceau depuis.
        \item La santé financière des partis dépend fortement des résultats aux dernières élections (flagrant pour le PS et LR).
        \item \alert{Fort endettement} : 110M\euro\ en 2012 pour LR (en baisse depuis). $\approx$ 30M\euro\ PS même au pouvoir. FN en hausse
    \end{itemize}
\end{frame}

\begin{frame}{Belgique}
    \begin{itemize}
        \item \alert{Principes} (Pays ayant poussé le plus loin le financement public) :
        \begin{itemize}
            \item Dépenses électorales limitées (1989).
            \item Limites très fortes aux dons privés.
            \item Une dotation aux partis chaque mois si représenté par au moins un élu dans l'une des deux assemblées du Parlement :
            \begin{itemize}
                \item 175K\euro\ si uniquement à la Chambre, 245K\euro\ si au Sénat aussi.
                \item Une subvention en fonction du nombre de voix (2.99\euro\ par voix ou 4.18\euro\ par voix selon la condition précédente) lors des dernières élections à la chambre. A comparer aux 1.42\euro\ en France et 1\euro\ en Allemagne.
                \item Une aide en fonction du nombre de parlementaires (60K\euro\ pour la Chambre, 22K\euro\ pour le Sénat).
            \end{itemize}
        \end{itemize}
        \item \alert{Constat} : Financement public direct aux partis très élevé si rapporté en pa. Seul comparable l'Espagne. Cela représente 2/3 des financements de certains partis (PS, Ecolos, Mouvement réformateur).
        \item \alert{Défauts} : Pas de remboursement des dépenses de campagnes, pas idyllique non plus cf. scandales politique-argent ou difficultés à former des gouv.
    \end{itemize}
\end{frame}

\begin{frame}{Allemagne}
    \begin{itemize}
        \item  \alert{Pays relativement généreux et précurseur en fonds publics} : 1954 financement indirect via remise fiscale sur don, 1959 financement direct introduit mais non stable. 1967 stabilisation sur base de remboursement des frais de campagne en fonction des voix obtenues. 1994 financement annuel des partis.
        \item \alert{Aujourd'hui} : si suffisamment de voix subvention annuelle au prorata, 0.45\euro\ de financement public pour 1\euro\ de don privé limité à 3300\euro. Au final ~160M\euro\ ou 2.39\euro\ pa stable depuis 2002.
        \item \alert{Elément révélateur} : 1958, déduction fiscale jugée non constitutionnelle (viol de l'égalité d'opportunité des différents partis) $\rightarrow$ introduction du financement direct.
        \item  \alert{Limites des financements publics} : pas + de la moitié des financements des partis (régulièrement atteint). Montant limite tous partis confondus déterminé chaque année.
        \item \alert{Non limite} des fonds privés
    \end{itemize}
\end{frame}

\section{Remboursement des dépenses de campagne}
\begin{frame}{France}
    \begin{itemize}
        \item \alert{fin 80} : si 5\% des suffrages $\rightarrow$ 47.5\% des dépenses de campagne remboursées.
        \item Montant très variable selon les types d'élections et les années. Les plus chères sont les municipales suivi des législatives.
        \item \alert{2012-2016} : 260M\euro\ 
    \end{itemize}
\end{frame}

\begin{frame}{Comparaisons internationales}
    \begin{itemize}
        \item \alert{Espagne} : financement direct aux partis très élevés, sur la même période 53M\euro\ par an 50\% de plus ramené en pa.
        \item \alert{Canada} : Manque de données mais $\approx$ 40M\euro\ sur chaque élection législative.
    \end{itemize}
\end{frame}

\begin{frame}{Synthèse}
    \begin{itemize}
        \item Avoir en tête les deux mécanismes de financement partis et campagnes qui sont souvent des vases communicants.
        \item {France} : en additionnant les deux 119M\euro\ ou 2.32\euro\ pa similaire à l'Allemagne. Attention au lieu commun sur la dotation supérieure des partis allemands donc.
        \item La vraie diff entre France et Allemagne c'est le financement des fondations qui sont liées aux partis en Allemagne et financées à hauteur de 7.55\euro pa pour rien en France.
        \item \alert{Thèse} : Principale faiblesse en France n'est pas le niveau de financement mais le mode d'allocation de celui-ci qui \alert{fige la démocratie}. Problème similaire en Allemagne. 
    \end{itemize}
\end{frame}

\section{Un système qui fige la démocratie}
\begin{frame}{Constat}
    \begin{itemize}
        \item \alert{Principale faiblesse} : la base quinquennale qui appauvrit les partis qui perdent les élections et enrichir ceux qui les gagnent.
        \item \alert{Conséquences} : pas tant un problème que ceux perdent n'aient plus d'argent mais cela empêche l'émergence de nouvelles forces sauf si soutenues par de forts investissements privés (car il est de toute évidence beaucoup plus facile de lever des fonds privés pour un mouvement qui promet la suppression de l'ISF... que pour une force politique qui se battrait pour l'augmentation des taux marginaux.)
        \item \alert{Anecdote marrante} : maison de la Mutualité 25K\euro\ pour En marche, 43K\euro pour les socialistes. wasted XD.
        \item \alert{Correlations troublantes (ou spurious)} : apparition en même temps d'En marche et nuit debout ou occupy wall street, le premier soutenu par des invest go président, les seconds disparaissent en quelques jours.
    \end{itemize}
\end{frame}
\begin{frame}{Explications et mesures correctives}
    \begin{itemize}
        \item \alert{Raison} : parce qu'il y en a qui doivent bosser (dixit) $\rightarrow$ exclusion de la vie politique. Comment monter un parti quand on a pas de financement privé et qu'il faut avoir gagner des élections pour avoir un financement public.
        \item Sauf si on a plein de blé cf Tea Party
        \item \alert{Mesure des BED} : Chaque année l'ensemble des citoyens décide comment allouer les financements publics aux différents mouvements politiques (y compris les plus récents). Inspiré du 2 pour mille italien et du fond présidentiel américain. Chaque citoyen alloue un montant fixe et identique d'argent public à un parti de son choix.
        \item \alert{Urgence} car sinon en suivant la tendance actuelle on se tourne vers le financement privé.
    \end{itemize}
\end{frame}

\section{Critique de moi :)}
\begin{frame}
    \begin{itemize}
        \item OMG c'était long pour dire 2 trucs :
        \begin{itemize}
            \item C'est chelou dans tous les pays
            \item On fait les BED les gars parce que ça a l'air d'être une bonne idée.
        \end{itemize}
        \item Sinon je suis plutôt d'accord avec sa proposition mais toujours sceptique de la prémisse 1 voix = 1 voix. On pourrait imaginer un BED avec une sorte d'attribution au mérite ou un truc du genre sans tomber dans l'élitisme.
        \item Par contre tout est mieux que l'argent privé donc go. Je pige pas comment c'est possible de pas déjà avoir interdit les dons privés.
        \item Sinon je trouve que les dépenses de campagne et des partis sont hautes pour rien. Dans ma tête les mecs ont un texte à écrire et à rendre disponible à la population. Il est peut-être temps de revenir à un mode de communication frugal et peut-être un peu standardisé afin que le fond soit le centre du combat.
        \item Enfin les dynamiques actuelles sont assez inquiétantes.
    \end{itemize}
\end{frame}
\end{document}
